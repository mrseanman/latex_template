% !TeX root = ../../main.tex
% \documentclass[class=guthesis, crop=false]{standalone}
% \begin{document}

\section{The conjecture and the objects involved}
Coxeter groups emerge as generalisations of reflection groups. A Coxeter group is defined by a particular group presentation. The data of this presentation is typically encoded by a labelled graph. The group $W$, coupled with the data of its presentation is called a Coxeter system, denoted $(W,S)$ where $S$ is the generating set of $W$. Given a Coxeter system $(W,S)$, we can construct a different group $G_W$, called the Artin group associated to $W$.

For affine Coxeter groups $W$, the configuration space $Y_W$ can be derived from a geometric realisation of $W$ as a subgroup of $\isom(\E)$, the group of isometries on a Euclidean space $\E$. We will consider $\E$ as $\R^n$ without the notion of origin. Specifically, $W$ is realised as a subgroup generated by a finite set of affine reflections $S$. Within $W$, we consider the set of all reflections $R$ (not necessarily finite). To each reflection $r \in R$ there is a corresponding codimension--1 space $H_r \subset \E$ that is the plane of reflection of $r$. We call such spaces hyperplanes. Note there is no requirement that these hyperplanes be subspaces of $\R^n$.

The configuration space is realised as the complement of the complexification of all such hyperplanes $H_r$. It is known by work of Brieskorn \cite{brieskorn_fundamentalgruppe_1971} that the fundamental group of $Y_W$ is $G_W$. Thus, proving the $K(\pi,1)$ involves showing that the higher homotopy groups of $Y_W$ are trivial. By previous work by Salvetti \cite{salvetti_topology_1987,salvetti_homotopy_1994}, there is a CW--complex $X_W$ called the Salvetti complex that is homotopy equivalent to $Y_W$. Showing homotopy equivalence to $X_W$ thus shows homotopy equivalence to $Y_W$. Because of this, the Salvetti complex is the starting point in a chain of homotopy equivalences reviewed in this work.

The finishing point of this chain is the interval complex $K_W$. This is a space realised using a certain poset structure on subsets of $W$. To this poset structure there is an associated group called the dual Artin group, denoted $W_w$. It was already known (by a now standard construction due to Garside \cite{garside_braid_1969}, extended by other authors, see \cite{charney_etal_bestvina_2002}) that $K_W$ was a classifying space for the dual Artin group for finite $W$. In \cite{paolini_salvetti_kpi1_2021}, the authors extend this result to affine $W$. Thus, showing $Y_W \simeq K_W$ for affine $W$ shows that (for affine $W$) the higher homotopy groups of $Y_W$ are trivial and that $W_w \cong G_W$.

In the following section, we will identify the intermediate spaces used in proving $X_W \simeq K_W$.

\section{Proof overview}

Here we will compile several main results from \cite{paolini_salvetti_kpi1_2021} in to two theorems. The concern of this work is \cref{thm:proof_overview} which proves that the \emph{Salvetti complex} $X_W$ is homotopy equivalent to the \emph{interval complex} $K_W$. A \emph{Coxeter element} is a non--repeating product of all the elements of $S$. A choice of order on $S$ corresponds to a choice of Coxeter element. Constructing an interval complex associated to $(W,S)$ involves making such a choice of Coxeter element $w \in W$.

For a subset $T \subseteq S$, the \emph{parabolic subgroup} $W_T$ is the subgroup of $W$ generated only by elements of $T$ and only with relations explicitly involving elements of $T$. A \emph{parabolic Coxeter element} $w_T$ is a product of all elements of $T$ that respects the order of multiplication in a Coxeter element $w \in W$. The space $X_W^\prime$ is a subspace of $K_W$ associated to parabolic Coxeter elements $w_T$ with $T \subseteq S$ such that $W_T$ is finite. Cells in $X_W$ also correspond to such subsets, which is used in proving $X_W \simeq X_W^\prime$.

The space $K_W^\prime$ is also a subspace of $K_W$. Given a CW--complex $X$, we can encode some information of how cells of $X$ attach to each other in a poset called the \emph{face poset} of $X$, denoted $\mathcal{F}(X)$. Connected components of preimages $\eta^{-1}(d)$ of a certain poset map $\eta \colon K_W \to \N$ have a linear structure as subposets of $\mathcal{F}(K_W)$. For each element $x \in \eta^{-1}(d)$, whether $x$ is in $K_W^\prime$ or not is determined based on whether $x$ comes in between two elements of $X_W^\prime$ in the linear structure of $\eta^{-1}(d)$.

\begin{theorem}[\hspace{1sp}{\cite{paolini_salvetti_kpi1_2021}}]
	Given an affine Coxeter system $(W,S)$, the configuration space $Y_W$ is homotopy equivalent to the order complex $K_W$.
	\label{thm:proof_overview}
\end{theorem}
\begin{proof}
	By \cref{thm:salvetti_hom_equiv_config} the Salvetti complex $X_W$ is homotopy equivalent to the configuration space $Y_W$. Therefore, we need only show $K_W \simeq X_W$. This is done through a composition of homotopy equivalences
	\begin{equation}
		X_W \labelrel\simeq{eqmid:salvetti_salvettiprime}
		X^\prime_W \labelrel\simeq{eqmid:salvettiprime_intervalcomplexprime}
		K^\prime_W \labelrel\simeq{eqmid:intervalcomplexprime_intervalcomplex}
		K_W
		\label{eqn:proof_overview}
	\end{equation}

	Where the results are gathered from the following sources:

	\begin{enumerate}[(a)]
		\item \cref{thm:salvetti_cx_equiv_X_prime} \cite[Theorem 5.5]{paolini_salvetti_kpi1_2021}
		\item \cref{thm:K_prime_hom_equiv_X_prime} \cite[Theorem 8.14]{paolini_salvetti_kpi1_2021}
		\item \cref{thm:subcomplex_K_prime_hom_equiv_K} \cite[Theorem 7.9]{paolini_salvetti_kpi1_2021}
	\end{enumerate}
	\vspace{-3em}
\end{proof}
\vspace{1.5em}

In \cite{paolini_salvetti_kpi1_2021}, another main result is shown.

\begin{theorem}[\hspace{1sp}{\cite[Theorem 6.6]{paolini_salvetti_kpi1_2021}}]
	Given an affine Coxeter system $(W,S)$, corresponding affine type Artin group $G_W$ and Coxeter element $w\in W$, the complex $K_W$ is a classifying space for the dual Artin group $W_w$.
	\label{thm:KW_classifyingspace}
\end{theorem}

We have that $\pi_1(Y_W) \cong G_W$ by \cite{brieskorn_fundamentalgruppe_1971}. Thus, considering $\pi_1(Y_W)$ and combining \cref{thm:proof_overview,thm:KW_classifyingspace} gives
\begin{align*}
	Y_W & \simeq K(G_W,1) \\
	G_W & \cong W_w
	\label{eq:artin_iso_dual}
\end{align*}
for affine $G_W$.

This proves the $K(\pi, 1)$ conjecture for affine Artin groups and provides a new proof than an affine Artin group is naturally isomorphic to its dual, which was already known for spherical or finite type Artin groups \cite{bessis_dual_2003} and affine Artin groups \cite{mccammond_sulway_artin_2017}.

Non--complete alternatives to \cite{brieskorn_fundamentalgruppe_1971} are \cite{vietdung_fundamental_1983} and \cite{fox_neuwirth_braid_1962}, which show $\pi_1(Y_W) \cong G_W$ for affine Coxeter groups and Coxeter groups of type $A_n$ respectively.

% \end{document}
